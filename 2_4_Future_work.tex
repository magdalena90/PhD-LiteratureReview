
\subsection{Future work}\label{FutureWork}
The work that will be done during the following months will be centred on two components: data integration and methodology, working on both simultaneously.

On the part of data integration, the most suitable datasets selected in Section \ref{Data} will be downloaded, explored and preprocessed (if necessary), an assessment will be made about the normalisation pipeline for the PEC Capstone Collection to decide if the normalised data is suitable to work with or if the raw data should be requested instead, and, based on this, the possibility of combining both Gandal and PEC Capstone Collection datasets will be considered. After this, openly available datasets from other data modalities will be researched and an evaluation will be made about their integration into a multiomics study.

On the methodology component, different modularity detection methods will be tested, including K-means or some other straightforward method, WGCNA and different versions of Independent Component Analysis to select the best suited to the analysis we decide to perform, a deeper research will be done about the use of classification methods, probably with the help of Positive-Unlabelled learning for the identification of new genes and pathways related to autism, and the most promising models will be implemented with the datasets created on the data integration component. Finally, improvements will be made to the best performing methods and the possibility of designing a new method will be considered if it is believed that it could have a better performance than the ones originally tested.
