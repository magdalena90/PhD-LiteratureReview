\section{Literature by topic}\label{LiteratureAppendix}

The papers used to created Table \ref{table:LiteratureByTopic} were the following:

\subsection{Spatial}
\begin{itemize}
    \item An anatomically comprehensive atlas of the adult human brain transcriptome \cite{hawrylycz_anatomically_2012}
    \item A Transcriptomic Atlas of Mouse Neocortical Layers \cite{belgard_transcriptomic_2011}
\end{itemize}

\subsection{Temporal}
\begin{itemize}
    \item Transcriptional landscape of the prenatal human brain \cite{miller_transcriptional_2014}
    \item Temporal dynamics and genetic control of transcription in the human prefrontal cortex \cite{colantuoni_temporal_2011}
    \item Transcriptome and epigenome landscape of human cortical development modeled in organoids \cite{amiri_transcriptome_2018}
\end{itemize}

\subsection{Cross-species}
\begin{itemize}
    \item Conserved cell types with divergent features between human and mouse cortex \cite{hodge_conserved_2018}
\end{itemize}

\subsection{ASD}
\begin{itemize}
    \item Postmortem brain abnormalities of the glutamate neurotransmitter system in autism \cite{purcell_postmortem_2001}
    \item Transcriptomic analysis of autistic brain reveals convergent molecular pathology \cite{voineagu_transcriptomic_2011}
    \item Transcriptome analysis reveals dysregulation of innate immune response genes and neuronal activity-dependent genes in autism \cite{gupta_transcriptome_2014}
    \item Immune transcriptome alterations in the temporal cortex of subjects with autism \cite{garbett_immune_2008}
    \item Gene expression in human brain implicates sexually dimorphic pahways in autism spectrum disorders \cite{werling_gene_2016}
    \item Transcriptome-wide isoform-level dysregulation in ASD, schizophrenia, and bipolar disorder \cite{gandal_transcriptome-wide_2018}
    \item Comprehensive functional genomic resource and integrative model for the human brain \cite{wang_comprehensive_2018}
    \item The transcription factor POU3F2 regulates a gene coexpression network in brain tissue from patients with psychiatric disorders \cite{chen_transcription_2018}
    \item Aberrant expression of long noncoding RNAs in autistic brain \cite{ziats_aberrant_2013}
    \item Shared molecular neuropathology across major psychiatric disorders parallels polygenic overlap \cite{gandal_shared_2018}
\end{itemize}

\subsection{Spatiotemporal}
\begin{itemize}
    \item Functional and Evolutionary Insights Into Human Brain Development Through Global Transcriptome Analysis \cite{johnson_functional_2009}
    \item Laminar and temporal expression dynamics of coding and noncoding RNAs in the mouse neocortex \cite{fertuzinhos_laminar_2014}
\end{itemize}

\subsection{Spatial + cross-species}
\begin{itemize}
    \item An anatomically comprehensive atlas of the adult human brain transcriptome \cite{hawrylycz_anatomically_2012}
    \item Molecular and cellular reorganization of neural circuits in the human lineage \cite{sousa_molecular_2017}
    \item Comprehensive transcriptome analysis of neocortical layers in humans, chimpanzees and macaques \cite{he_comprehensive_2017}
\end{itemize}

\subsection{Spatiotemporal + ASD}
\begin{itemize}
    \item Spatio-temporal transcriptome of the human brain \cite{kang_spatio-temporal_2011}
\end{itemize}

\subsection{Temporal + ASD}
\begin{itemize}
    \item Age-dependent brain gene expression and copy number anomalies in autism suggest distinct pathological processes at young versus mature ages \cite{chow_age-dependent_2012}
\end{itemize}

\subsection{Spatiotemporal + cross-species + ASD}
\begin{itemize}
    \item A comprehensive transcriptional map of primate brain development \cite{bakken_comprehensive_2016}
    \item Integrative functional genomic analysis of human brain development and neuropsychiatric risk \cite{li_integrative_2018}
    \item Spatiotemporal transcriptomic divergence across human and macaque brain development \cite{zhu_spatiotemporal_2018}
\end{itemize}
